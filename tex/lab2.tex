\documentclass[12pt, letterpaper]{article}
\usepackage[utf8]{inputenc}
\usepackage[margin=1in]{geometry}
\usepackage{amsmath, amssymb, hyperref, graphicx, listings}
\lstset{basicstyle=\ttfamily\small, breaklines=true}

\title{Robotics - Lab 2}
\author{Taiming YuenJames}

\begin{document}

\maketitle

\section{Link and Arm Lengths}

Based on the xacro file provided, the link lengths add up to a total reach of approximately 1.0872 meters.

\[0.425+0.39225+0.093+0.09465+0.0823=1.0872\]

\section{Maximum Range}

In the x direction, I am able to achieve a maximum range of 0.948992457 meters. In the y direction, I am able to achieve a maximum range of 0.948989495 meters. In the z direction, I am able to achieve a maximum range of 1.03122057 meters. I am not able to reach the theoretical maximum range exactly because the points I chose (e.g. [0, 0, 2]) are optimally solved diagonally (with respect to the base), yeilding a shorter maximum length.

\section*{Code}

\lstinputlisting[language=Python]{../lab2.py}

\section*{Visualizations}

\begin{figure}[h]
	\centering
	\includegraphics[width=1\textwidth]{sphere.png}
	\caption{Reachable Poses as Convex Hull}
\end{figure}

\begin{figure}
	\centering
	\includegraphics[width=1\textwidth]{fail.png}
	\caption{Collision Checking Test Case Failure}
\end{figure}

\begin{figure}
	\centering
	\includegraphics[width=1\textwidth]{pass.png}
	\caption{Collision Checking Test Case Pass}
\end{figure}

\end{document}